%Titelseite

\definecolor{hsu-rot}{rgb}{0.6745,0.0118,0.2275}
\definecolor{hsu-grau}{rgb}{0.5608,0.5569,0.549}
\begin{titlepage}
\hfill\includegraphics[width=5cm]{hsu_4c}
% latex erwartet .eps-Datei pdflatex erwartet .pdf-Datei
\vspace{5cm}\\
\color{hsu-rot}
\large\bfseries
Henning Wolf\\[2ex]
\color{hsu-grau}
Tutorial zur Erstellung von verschiedenen Soft Core Prozessoren\\[2ex]

%{\small\it Im engl. Titel das erste Wort sowie alle weiteren Worte außer Artikel, Partikel und
%  Präpositionen groß schreiben. Bitte bei der Formulierung und
 % Orthografie des Titel sorgfältig sein. Er wird von hier in das Zeugnis
  %übernommen!}
% An erster Stelle, falls die ganze Arbeit in Englisch geschrieben wurde
% Das Prüfungsamt benötigt den englischen Titel, da das Zeugnis auch
% in einer englischen Version ausgefertigt wird.
\vfill
Studien-Arbeit % Mit Bindestrich
%Studienarbeit   % Ohne Bindestrich
%Master-Arbeit   % Mit Bindestrich
\normalsize
\vspace{0.5cm}
\begin{center}
\mbox{\rule{-0.13\paperwidth}{0ex}\color{hsu-rot}{\rule{1.05\paperwidth}{2mm}} }
\end{center}
\vspace{1cm}
\color{hsu-grau}
Fakultät für Elektrotechnik
\begin{tabbing}
Weiterer Prüfer: \= \kill
Studiengang:\>
Informatikingenieurswesen %(bei Bachelor-Arbeit)
%Elektrische Energietechnik %(ggf. bei Master-Arbeit)
%Informationstechnik %(ggf. bei Master-Arbeit)
\\
Matr.-Nr.\> 874320 / ET2014 \\
%{\small\it (Entfällt bei Studienarbeit)}  % Bitte auskommentieren
\\
Betreuer:\>Univ.-Prof. Dr. phil. nat. habil. Bernd Klauer
%\\{\small\it (hier betreuenden Prüfer, nicht wiss. Mitarbeiter eintragen )} % Bitte auskommentieren
%{\small\it (Entfällt bei Studienarbeit)}  % Bitte auskommentieren
\end{tabbing}
\vspace{-2.5cm}
\end{titlepage}
\pagenumbering{roman}
\cleardoublepage

\section*{Erkl"arung}
% Text nicht ändern, er muss der SPO entsprechen
Hiermit versichere ich, dass ich diese Arbeit selbstst"andig verfasst,
keine anderen als die im Quellen- und Literaturverzeichnis genannten Quellen
und Hilfsmittel, insbesondere keine dort nicht genannten Internet-Quellen
benutzt,
alle aus Quellen und Literatur wörtlich oder sinngemäß entnommenen Stellen als
solche kenntlich gemacht habe und dass die auf einem elektronischen
Speichermedium abgegebene Fassung der Arbeit der gedruckten entspricht.
\vspace{1.5cm}\\
Hamburg,\\[-1ex]
\mbox{}\hspace{2.2cm}\parbox[t]{4cm}{\centering \dotfill\\(Datum)}\hspace{0.5cm}
                    \parbox[t]{8cm}{\centering \dotfill\\(Unterschrift)}
\cleardoublepage


\cleardoublepage
\tableofcontents
\cleardoublepage
