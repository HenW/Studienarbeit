\chapter{Zusammenfassung und Ausblick}\label{ch:schluss}

In dieser Arbeit wird beschrieben, wie es ermöglicht wurde, per \ac{gps} und mithilfe des Beschleunigungssensors
ADXL 362 eine Positionsbestimmung innerhalb des PRHS-Framework durchzuführen. Dazu wurden zuerst sämtliche, einzubindende
Komponenten ausführlich beschrieben und dann auf der Hardwareseite angebunden. \\
Der \ac{gps}-Sensor wurde hierbei an einen sogenannten Pmod-Port angeschlossen und die vorhandene Pinkonfiguration
in das System eingetragen. Des Weiteren wurde für den \ac{gps}-Sensor ein Modul entwickelt,
welches per \ac{uart}-Schnittstelle mit dem System kommunizieren kann. In diesem Modul wurden zwei Untermodule
vereint und an den \ac{prhs}-Bus gekoppelt. Um nun die Kommunikation zwischen diesem Modul und dem System
sicherzustellen, wurde ein Treiber entwickelt, der es ermöglicht, die vom Sensor gesendeten Daten zu verstehen
und etwaigen Applikationen zur Verfügung zu stellen. Dabei kam unter anderem der \ac{gpsd} zum Einsatz, der die
Daten dem Betriebssystem, bereits im richtigen Format, zur Verfügung stellt.\\
Der Beschleunigungssensor, welcher sich bereits auf dem \ac{fpga}-Board befindet, wurde per
\ac{spi}-Schnittstelle angebunden. Es wird ein neues \ac{spi}-Modul angelegt und implementiert. Für
dieses wird ebenfalls ein Treiber entwickelt, welcher die Kommunikation herstellt. So kann das
System die Daten empfangen und interpretieren. Nachdem Erprobungen für die einzelnen Sensoren getätigt wurden,
wurde eine Demo-Applikation entwickelt, welche dem Nutzer per \ac{tui} die \ac{gps}- und Beschleunigungsdaten darstellt.\\
Es wurde erreicht, dass der Nutzer des Systems die \ac{gps}-Daten, wie zum Beispiel Längen- und Breitengrad,
sowie die Höhe über Normalnull und Geschwindigkeit, nicht nur in der Applikation einsehen, sondern sie auch
über den angelegten Deviceport weiter nutzen kann. Die Möglichkeit bietet der Beschleunigungssensor ebenfalls, welcher
neben den Daten der drei Achsen, zusätzlich die Temperatur anzeigt und diese für weitere Anwendungen nutzbar sind.\\
So können diese Daten in der Zukunft in sämtlichen Programmen und Anwendungen genutzt werden, wie zum Beispiel zur
Lokalisierung eines autonomen Systems. Um das gesamte Potenzial des Systems zu nutzen ist eine Weiterentwicklung
möglich und nötig.\\
Der \ac{gps}-Sensor, als auch der Beschleunigungssensor bieten die Möglichkeit weitere
Daten und Parameter zu empfangen und weiterzuverarbeiten. So stellt der \ac{gps}-Sensor, neben den oben genannten
und in der Applikation angezeigten Daten, etliche weitere Parameter zur Verfügung. Diese sind im Anhang in Form
des Datenblattes hinterlegt. Das bedeutet, dass sowohl der Umfang der Applikation, als auch die Einsatzgebiete
dieser Sensoren deutlich vergrößert werden können.
