\chapter{Zusammenfassung und Ausblick}\label{ch:schluss}

In dieser Arbeit wurden grundsätzliche Themen, welche mit der SoftCore-Entwicklung zusammenhängen in Form der Grundlagen erläutert und zum Teil,
 zum besseren Verständnis, grafisch dargestellt.\\
Des Weiteren wurden die drei genannten Systeme näher erläutert und es wurde Schritt für Schritt erklärt, wie diese
Systeme angepasst werden müssen, um eine Ausführung auf dem Digilent Nexys4 DDR \ac{fpga}-Board zu ermöglichen.
Ebenfalls wurde beschrieben wie ein, zu den einzelnen Systemen passendes, Betriebssystem konfiguriert werden muss.\\
Die Eigenschaften der drei SoftCores wurden jeweils kurz erläutert um eine gewissen Vergleichbarkeit herzustellen.
Diese Vergleichbarkeit ist es, die eine auf dieser Arbeit basierende Betrachtung interessant macht. Neben den drei Systemen gibt es
an der Professur für Technische Informatik an der Helmut-Schmidt-Universität / Universität Hamburg der Bundeswehr ein weiteres System,
das sogenannte \ac{prhs}. Für dieses Framework steht ebenfalls ein Betriebssystem zur Verfügung, das \ac{l4prhs}. So würde es sich durchaus anbieten
, auf Grund dieser Arbeit und dem hier an der Universtät vorhandenen Systems, einen Vergleich dieser Modell auf verschiedenen Ebenen durchzuführen. Mit Hilfe
einer Benchmark könnte eine Analyse der Systeme, im Bezug auf Ergebnisse mit vorher festgelegtem Bezugswert, erstellt werden und in einer weiteren Arbeit festgehalten werden.
