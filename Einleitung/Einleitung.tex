% \chapter{Einleitung}\label{ch:einleitung}
%
% Die Technologieentwicklung, im speziellen die der Computer, kann auf eine rasante Entwicklung in den letzten
% Jahrzehnten zurückblicken. Jedoch ist diese Technik mittlerweile so hochentwickelt, dass es mit zunehmender Zeit
% schwieriger wird, die Bauteile noch schneller, kleiner und effizienter herzustellen, denn auch hier gibt es Grenzen.
% So liegt es an den Entwicklern neue Wege zu finden, um diese Beschränkungen zu umgehen. \\
% So kam es zur Entwicklung der \acp{fpga}, welche es nun ermöglichen sollten, komplexe Logikschaltungen
% rekonfigurierbar herzustellen, sodass diese jederzeit erneuert, verändert und damit den Gegebenheiten und
% Anforderungen angepasst werden konnten. Durch die Optimierung während der Laufzeit können Kosten gesenkt und
% Ressourcen geschont werden. Mittlerweile sind moderne \acp{fpga} mit genügend Recourcen ausgestattet um eine
% große Anzahl an Logikgattern zu berechnen und sind in der Lage komplexe Rechenoperationen durchzuführen.\\
% Genau diese Entwicklung nutzte die Professur für Technische Informatik an der
% Helmut-Schmidt-Universität / Universität Hamburg der Bundeswehr, um im Rahmen dieser ein eigenes
% System zu entwickeln und auf dem Gebiet zu forschen. Das dort genutzte \ac{prhs}-Framework ermöglicht es,
% eine Nachbildung von Rechnern auf vielen verschiedenen \ac{fpga}-Boards zu nutzen. Neben den
% verschiedenen Hardwaremodulen wird als Betriebssystem das sogenannte \ac{l4prhs} genutzt.
%
% \section{Ziel der Arbeit}\label{kap:zielderarbeit}
%
% Das Ziel dieser Arbeit ist es, dieses vorhandene System um zwei weitere Module zu erweitern und damit
% eine Positionsbestimmung mit dem \ac{prhs}-Framework zu ermöglichen. Diese Module sind zum einen
% ein \emph{Pmod \ac{gps}-Receiver} und zum anderen der On-Board Beschleunigungssensor des Nexys 4 DDR. Für die
% Umsetzung wird die Hardware mithilfe der Hardwarebeschreibungssprache \ac{vhdl} beschrieben und der Treiber für Linux
% in der Programmiersprache \emph{C}. Für die Nutzung und Steuerung der
% Sensoren werden, die im \ac{prhs}-Framework bereits vorhandenen, Schnittstellen genutzt und an die Anforderungen angepasst.
%
% \section{Struktur der Arbeit}\label{kap:strukturderarbeit}
%
% Zur Umsetzung des Zieles, wird die Arbeit wie folgt gegliedert.\\
% Zu Beginn der Arbeit werden im Kapitel~\ref{kap:grundlagen} die theoretischen Grundlagen vermittelt. Als
% Ausgangspunkt wird im Kapitel~\ref{kap:fpga} auf den Aufbau und die Funktionsweise von \acp{fpga} eingegangen,
% gefolgt von einer näheren Erklärung des \ac{prhs}-Framework in Kapitel~\ref{kap:prhs}. Die Kommunikation zwischen
% Peripheriegeräten und dem System findet über Schnittstellen statt (Kapitel~\ref{kap:schnittstellen}).
%  Dies beinhaltet die im Rahmen dieser Arbeit genutzten Schnittstellen \ac{uart}, für den \ac{gps}-Receiver, in Kapitel~\ref{kap:uart}
%   und die \ac{spi}-Schnittstelle in Kapitel~\ref{kap:spi} für den Beschleunigungssensor ADXL362.
% Die technischen Daten und wichtige Informationen des genutzten \ac{gps}-Sensors werden in
% Kapitel~\ref{kap:gpsreceiver} erläutert, wobei im Unterkapitel~\ref{kap:gpsd} explizit auf den
% sogenannten \ac{gpsd} eingegangen wird. Im Kapitel~\ref{kap:accelerometer} werden die Funktionsweisen und
% Eigenschaften des Beschleunigungssensor ADXL362 dargestellt. Das Kapitel~\ref{kap:linux} erklärt grob die
% Funktionsweise eines Betriebssystems, wie hier anhand des Beispiels Linux. Hierbei gibt es zwei Unterkapitel.
% Das Kapitel~\ref{kap:treiber}, welches die wesentlichen Eigenschaften und Funktionsweisen
% eines Treibers erläutert und das Kapitel~\ref{kap:buildroot}. Dieses beschreibt ein Tool, welches zu Erzeugung
% eines Betriebssystems genutzt werden kann.\\
% Die praktische Implementierung in Kapitel~\ref{ch:praktischeImplementierung} gliedert sich in zwei Teile.
% Ein Teil davon besteht aus dem \ac{gps}-Sensor, dessen Hardwareanbindung in Kapitel~\ref{kap:gpssensor} näher
% erklärt wird und einer Beschreibung des Treibers in Kapitel~\ref{kap:gpstreiber}. Der zweite Teil von Kapitel~\ref{ch:praktischeImplementierung}
%  beschäftigt sich mit der Anbindung des Beschleunigungssensors in Kapitel~\ref{kap:implementierungaccel}, welche wiederrum aus
% der Beschreibung der Hardwareanbindung in Kapitel~\ref{kap:accelimplementierung} und der des Treibers
% in Kapitel~\ref{kap:acceltreiber} besteht. \\
% Um nun die erarbeiteten Ergebnisse zu testen, wird in Kapitel~\ref{kap:erprobung} beschrieben, wie erst
% der \ac{gps}-Sensor in Kapitel~\ref{kap:gpserprobung} und der Beschleunigungssensor in Kapitel~\ref{kap:accelerprobung} getestet wird.
% Abschließend werden alle Module in einer Applikation zusammengefasst,
% welche in Kapitel~\ref{kap:applikation} beschrieben wird.\\
% Im Kapitel~\ref{ch:schluss} wird die Arbeit noch einmal zusammengefasst und ein Ausblick gegeben.\\
