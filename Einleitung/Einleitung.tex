\chapter{Einleitung}\label{ch:einleitung}

Die Technologieentwicklung, im speziellen die der Computer, kann auf eine rasante Entwicklung in den letzten
Jahrzehnten zurückblicken. Jedoch ist diese Technik mittlerweile so hochentwickelt, dass es mit zunehmender Zeit
schwieriger wird, die Bauteile noch schneller, kleiner und effizienter herzustellen, denn auch hier gibt es Grenzen.
So liegt es an den Entwicklern neue Wege zu finden, um diese Beschränkungen zu umgehen. \\
So kam es zur Entwicklung der \acp{fpga}, welche es nun ermöglichen sollten, komplexe Logikschaltungen
rekonfigurierbar herzustellen, sodass diese jederzeit verändert und damit den Gegebenheiten und
Anforderungen angepasst werden konnten. Durch die Optimierung während der Laufzeit können Kosten gesenkt und
Ressourcen geschont werden. Mittlerweile sind moderne \acp{fpga} mit genügend Recourcen ausgestattet um eine
große Anzahl an Logikgattern zu berechnen und sind in der Lage komplexe Rechenoperationen durchzuführen.\\

\section{Ziel der Arbeit}\label{kap:zielderarbeit}

Das Ziel dieser Arbeit ist es, verschiedene Soft-Prozessoren zu betrachten, diese an das Digilent Nexys4 DDR \ac{fpga}-Board anzupassen und zu synthetisieren. Etwaige Fehler wurden behoben,
welche auf Grund der OpenSource-Projekt durchaus vorkamen. Behandelt wurden in diesem Fall die OpenSource-SoftCores \emph{lowRISC}, welcher auf der RISC-V Struktur basiert,
 sowie der \emph{LEON3}-Prozessor,
der auf der SPARC V8 Architektur aufgebaut ist. Des Weiteren wurde der, vom Hersteller des Boards angebotene, MicroBlaze konfiguriert und synthetisiert.\\
Neben der Hardware sollte auf jedem der Systeme das Betriebsystem Linux ausgeführt werden, um einen realistischen Vergleich herzustellen.


 \section{Struktur der Arbeit}\label{kap:strukturderarbeit}

Zur Beschreibung des Vorgehens, wird die Arbeit wie folgt gegliedert.\\
 Zu Beginn der Arbeit werden im Kapitel \ref{kap:grundlagen} die theoretischen Grundlagen vermittelt. Als
Ausgangspunkt wird im Kapitel~\ref{kap:softcoreprozessoren} auf die Definition und den Unterschied zwischen SoftCore- beziehungsweise HardCore-Prozessoren eingegangen,
gefolgt von einer näheren Erklärung der \ac{mmu} in Kapitel~\ref{kap:mmu}. Die Kommunikation zwischen
Peripheriegeräten und dem System findet über Schnittstellen statt. Die mit diesem Thema zusammenhängenden
Unterkapitel befinden sich im Abschnitt(\ref{kap:peripherie}).
Dies beinhaltet die im Rahmen dieser Arbeit genutzte Schnittstelle \ac{uart}, welche zur Kommunikation mit dem
Zielsystem benötigt wird(\ref{kap:uart}). Ebenfalls sind die sogenannten Interrupts ein wichtiger
Bestandteil dieser Systeme, nähe beschrieben in Kapitel ~\ref{kap:interrupt}.
Auf Seiten der Software wurde näher auf Funktionsweise eines Compilers(\ref{kap:compiler}) eingegangen und in diesem Fall
besonders auf die verschiedenen Modi des Compilers(\ref{kap:compilermode}).
Das Kapitel~\ref{kap:linux} erklärt grob die
Funktionsweise eines Betriebssystems, wie hier anhand des Beispiels Linux. Hierbei gibt es drei Unterkapitel.
Das Kapitel~\ref{kap:treiber}, welches die wesentlichen Eigenschaften und Funktionsweisen
eines Treibers erläutert, gefolgt von dem Kapitel ~\ref{kap:devicetree} und dem Kapitel~\ref{kap:buildroot}.
Dieses beschreibt ein Tool, welches zu Erzeugung
eines Betriebssystems genutzt werden kann.\\
Die praktische Implementierung in Kapitel~\ref{kap:softcores} gliedert sich in drei Teile.
Im ersten Teil geht es um die Implementierung des Xilinx MicroBlaze-Prozessor(Kapitel~\ref{kap:microblaze}),
sowie die Ausführung des erzeugten Linuxsystems(\ref{kap:mcausführenlinux}).
In Kapitel~\ref{kap:lowrisc} wird Schritt für Schritt erklärt, wie das \emph{lowRISC}-System zu konfigurieren ist. Des Weiteren wird
in diesem Kapitel auf die Generierung des dazugehörigen Linux eingegangen. Ähnlich aufgebaut ist das Kapitel~\ref{kap:leon3}, in welchem
erst die Hardware(\ref{kap:leon3hardware}) des LEON3 und anschließend in Kapitel~\ref{kap:linuxleon} das passende Betriebssystem näher erläutert wird.\\
Im Kapitel~\ref{ch:schluss} wird die Arbeit noch einmal zusammengefasst und ein kurzer Ausblick gegeben.\\
