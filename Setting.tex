\documentclass[a4paper,12pt,ngerman]{scrreprt}
\usepackage[pdftitle={Studienarbeit Henning Wolf},colorlinks=true,linkcolor=blue,citecolor=blue,anchorcolor=blue,pagecolor=blue,plainpages=false,pdfpagelabels,backref=false]{hyperref}
\usepackage[german]{babel}
\usepackage{babelbib}
\usepackage{graphicx}
\usepackage{epstopdf}
\usepackage{lscape}
\usepackage{multicol}
\usepackage{amsmath}
\usepackage{amssymb}
\usepackage{amstext}
\usepackage{amsfonts}
\usepackage{mathrsfs}
\usepackage{esint}
\usepackage{amsthm}
\usepackage{url}
\usepackage{pifont}
\usepackage[utf8]{inputenc}
\usepackage[T1]{fontenc}
\usepackage[tight,footnotesize]{subfigure}
\usepackage[percent]{overpic}
\usepackage{listings}
\usepackage{booktabs}
\usepackage{multirow}
\usepackage{tikz}
\usepackage{framed}
\usepackage{pdfpages}
\usepackage{colortbl}
\usepackage{float}
%\usepackage {xcolor}
\usepackage {xcolor}
\usepackage{acronym}
\usepackage[onehalfspacing]{setspace}
\pagestyle{headings}
\newcommand{\im}{{\rm Im}}
\newcommand{\re}{{\rm Re}}
\newcommand{\grad}{{\rm grad}}
\newcommand{\dist}{{\rm dist}}
\newcommand{\diam}{{\rm diam}}
\newcommand{\const}{{\rm const}}
\newcommand{\rang}{{\rm rang\,}}
\newcommand{\Bild}{{\rm Bild}}
\newcommand{\diag}{{\rm diag}}
\newcommand{\graph}{{\rm graph\,}}
\newcommand{\DD}{{\rm D}}
\newcommand{\dd}{{\rm d}}
\newcommand{\ds}{\displaystyle}
\setlength{\parindent}{0em}
\newcommand{\N}{\mathbb{N}}
\newcommand{\Z}{\mathbb{Z}}
\newcommand{\Q}{\mathbb{Q}}
\newcommand{\R}{\mathbb{R}}
\newcommand{\C}{\mathbb{C}}
\newcommand{\K}{\mathbb{K}}
%%%%%%%%%%%%%%%%%%%%%%%%%
\renewcommand{\ae}{{\"a}}
\newcommand{\ue}{{\"u}}
\renewcommand{\oe}{{\"o}}
\newcommand{\Ae}{{\"A}}
\newcommand{\Ue}{{\"U}}
\newcommand{\Oe}{{\"O}}
%\newcommand{\ss}{{"s}}
%%%%%%%%%%%%%%%%%%%%%%%%%%%
\newcommand{\calE}{\mathcal{E}}
\newcommand{\calI}{\mathcal{I}}
\newcommand{\calL}{\mathcal{L}}
\newcommand{\calC}{\mathcal{C}}

\numberwithin{equation}{chapter}

\newtheoremstyle{style1}
  {0.5cm}              %Space above
  {0.5cm}              %Space below
  {\itshape}                      %Body font: original {\normalfont}
  {}                      %Indent amount (empty = no indent,%\parindent = paraindent)
  {\normalfont\bfseries}  %Thm head font original
  {}{\newline}
  {{\normalfont\bfseries \thmname{#1}\thmnumber{ #2}\thmnote{ (#3)}}}
\theoremstyle{style1}
%-----------------------------
\newtheorem{Sa}{Satz}[chapter]
\newtheorem{Ko}[Sa]{Folgerung}

\newtheoremstyle{style2}
  {0.5cm}
  {0.5cm}
  {}
  {}
  {\normalfont\bfseries}
  {}{\newline}
  {{\normalfont\bfseries \thmname{#1}\thmnumber{ #2}\thmnote{ (#3)}}}
\theoremstyle{style2}
\newtheorem{Def}[Sa]{Definition}
%\newtheorem*{Bsp}{Beispiel}
%\newtheorem*{Bspe}{Beispiele}
\newtheorem*{Bem}{Bemerkung}
\newtheorem*{Bemm}{Bemerkungen}

\newenvironment{Bsp}{\textbf{Beispiel.}\rmfamily }{\hspace{\fill} $\triangleleft$}
\newenvironment{Bspe}{\textbf{Beispiele.}\rmfamily }{\hspace{\fill} $\triangleleft$}

\newcommand{\lk}{\left \langle \,}
\newcommand{\rk}{\, \right \rangle}

\renewcommand{\footnoterule}{\vspace*{-3pt}\hrule height 0.4pt \vspace{2.6pt}}

\long\def\symbolfootnote[#1]#2{\begingroup \def\thefootnote{\fnsymbol{footnote}}\footnote[#1]{#2}\endgroup}

\newcommand{\V}[1]{\begin{pmatrix}#1\end{pmatrix}}
\lstloadlanguages{Matlab}
\lstnewenvironment{MatlabCode}[1][]
{\microtypesetup{activate=false}      %Captions im Programmcode
	\lstset{
		language= Matlab,
		basicstyle=\ttfamily,                     %generell Schreibmaschinenschrift
		basicstyle=\scriptsize,
		keywordstyle=\color{darkblue},
		commentstyle=\color{darkgreen},
		stringstyle=\color{string},
		backgroundcolor=\color{yellow},      %Hintergrundfarbe
		showstringspaces=false,                  %In Strings keine Backspace zeichen breaklines=true,
		captionpos=b,                                 %Beschriftungsposition
		numbers   =   left,                              %links Zeilennummern
		xleftmargin=.04\textwidth,
		%frame=single,                                 %shadowbox, leftline, lines, topline, t, r, b, l
		#1}
}
{}


% Beweisumgebung neu definieren:
\renewcommand\proof[1][\proofname]{
 \par
 \pushQED{\qed}
 \normalfont  \topsep 0pt plus 0pt \relax
 \trivlist
 \item[\hskip\labelsep
               \bfseries #1: ]\ignorespaces
 }
% % % % % % % % % % % % % % % % % % % % % % % % % % %
\lstset{ %
	language=Matlab,                % the language of the code
	basicstyle=\footnotesize,           % the size of the fonts that are used for the code
	numbers=left,                   % where to put the line-numbers
	numberstyle=\tiny\color{black},  % the style that is used for the line-numbers
	stepnumber=1,                   % the step between two line-numbers. If it's 1, each line
	% will be numbered
	numbersep=5pt,                  % how far the line-numbers are from the code
	backgroundcolor=\color{white},      % choose the background color. You must add \usepackage{color}
	showspaces=false,               % show spaces adding particular underscores
	showstringspaces=false,         % underline spaces within strings
	showtabs=false,                 % show tabs within strings adding particular underscores
	frame=single,                   % adds a frame around the code
	rulecolor=\color{black},        % if not set, the frame-color may be changed on line-breaks within not-black text (e.g. commens (green here))
	tabsize=2,                      % sets default tabsize to 2 spaces
	captionpos=b,                   % sets the caption-position to bottom
	breaklines=true,                % sets automatic line breaking
	breakatwhitespace=false,        % sets if automatic breaks should only happen at whitespace
	title=\lstname,                   % show the filename of files included with \lstinputlisting;
	% also try caption instead of title
	keywordstyle= \color{blue},          % keyword style
	commentstyle=\color{dkgreen},       % comment style
	stringstyle=\color{mauve},         % string literal style
	escapeinside={\%*}{*)},            % if you want to add LaTeX within your code
	morekeywords={*,...}               % if you want to add more keywords to the set
}
\definecolor{dkgreen}{rgb}{0,0.6,0}
\definecolor{gray}{rgb}{0.5,0.5,0.5}
\definecolor{mauve}{rgb}{0.58,0,0.82}
\definecolor{rot}{rgb}{0.75,0,0.26}
\definecolor{grun}{rgb}{0.75,0.44,0}
\definecolor{dgrun}{rgb}{0.0,0.44,0.42}
